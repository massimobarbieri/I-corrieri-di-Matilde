%\documentclass{article}{dndbook}
\documentclass[letterpaper,openany,nodeprecatedcode]{dndbook}

\usepackage[italian]{babel}
\usepackage{graphicx}% http://ctan.org/pkg/graphicx

\usepackage[T1]{fontenc}
\usepackage{frcursive}
\usepackage{calligra}
\newcommand{\setfont}[2]{{\fontfamily{#1}\selectfont #2}}
\begin{document}

\setfont{calligra}{Alla più nobile Grancontessa Matilde di Canossa.}



\setfont{calligra}{Mia Nobile Grancotessa di Toscana ho saputo che suo marito Goffredo III Duca della Bassa Lorena è recentemente tornato in Italia con l'intento di riconquistare la fiducia perduta dell'altissima Grancontessa. Sono a tale proposito stato informato che il Duca alloggia presso il Castello di Canossa.}

\setfont{calligra}{Mi duole tuttavia informarla, mia cara Grancontessa, che mi è giunta voce da fonti che non mi è permesso rivelare che Roberto I delle Fiandre sta tramando di uccidere il rispettabile Duca Goffredo.}

\setfont{calligra}{Sono certo che lei voglia che nulla accada al povero Duca, tuttavia la mia preoccupazione è rivolta a lei mia cara Grancontesse. Nella maleaugurata sorte che il Granduca perisca per mani di luridi manigoldi, non vorrei che l'altissima Grancontessa venisse accusata di aver tessuto trame contro suo marito il Duca.}

\setfont{calligra}{Le suggerisco pertanto, con la massima umiltà, di fare tutto quello che è in suo potere per proteggere il Duca durate il periodo del suo soggiorno in Italia. Sempre che non sia troppo tardi naturalmente}

\setfont{calligra}{Il suo umile servitore.}

\setfont{calligra}{Guelfo IV d'Este}

\begin{figure}
\includegraphics[width=5cm]{img/sigillo.png}
\end{figure}

\end{document}