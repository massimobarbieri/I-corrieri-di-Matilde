%\documentclass{article}{dndbook}
\documentclass[letterpaper,openany,nodeprecatedcode]{dndbook}

\usepackage[italian]{babel}
\usepackage{graphicx}% http://ctan.org/pkg/graphicx
\usepackage[T1]{fontenc}
\usepackage{frcursive}
\usepackage{calligra}
\newcommand{\setfont}[2]{{\fontfamily{#1}\selectfont #2}}
\begin{document}

\setfont{calligra}{Alla più nobile Grancontessa Matilde di Canossa.
}



\setfont{calligra}{Mia Nobile Grancotessa di Toscana

ho saputo che suo marito Goffredo III, Duca della Bassa Lorena, è recentemente tornato in Italia per riconquistare la fiducia perduta di Sua Altissima Grancontessa. Sono a tale proposito stato informato che in questi giorni fine anno il Duca alloggia presso il Castello di Canossa.}

\setfont{calligra}{Mi duole tuttavia informarla, mia adorata Grancontessa, che mi è giunta voce, da fonti che non mi è permesso rivelare, che Roberto I delle Fiandre sta tramando di uccidere il Duca Goffredo.}

\setfont{calligra}{Sono certo che la mia Nobile Grancontessa voglia che nulla accada al povero Duca, tuttavia la mia preoccupazione è rivolta a lei, mia cara Grancontessa. Nella maleaugurata sorte che il Granduca perisca per mani di luridi manigoldi, non vorrei che Sua Altissima Grancontessa venisse accusata di aver tessuto trame contro suo marito il Duca.}

\setfont{calligra}{Le suggerisco pertanto, con la massima umiltà, di fare tutto quello che è in suo potere per proteggere il Duca durate il periodo del suo soggiorno in Italia, non solo per il Duca, ma anche per garantire il prestigio e l'integrità della casata di cui la mia Nobilissima Grancontessa fa parte: la casata dei Canossa.}

\setfont{calligra}{Il suo umile servitore.}

\setfont{calligra}{Guelfo IV d'Este}

\begin{figure}
\includegraphics[width=3cm]{img/sigillo.png}
\end{figure}

\end{document}